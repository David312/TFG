
\chapter{Estado del Arte}
\label{chap:estado_del_arte}

El software del OBC del UPMSat-2 está desarrollado en el lenguaje Ada usando un
compilador cruzado para procesadores LEON3.

Los motivos para emplear este lenguaje de programación son bastantes, pero
principalmente destaca por ser un lenguaje orientado al desarrollo de software
para sistemas
empotrados y de tiempo real como es el UPMSat-2, ya que incorpora de forma
innata todos los mecanismos necesarios para el manejo y control de la
concurrencia. Además cabe destacar que el lenguaje Ada es
uno de los lenguajes oficiales de la Agencia Espacial Europea
(ESA\cite{web-ESA}) para este tipo de sistemas.

Puesto que cada uno de los subsistemas que componen el OBC tiene sus propias
tareas o hilos de ejecución para llevar a cabo su funcionalidad, es necesario
controlar de alguna manera sus prioridades y la compartición de recursos.
Para ello, se emplea el perfil de Ravenscar\cite{perfil-ravenscar},
que es un subconjunto de
las características de Ada que restringe ciertos aspectos como los siguientes:

\begin{itemize}
\item Una tarea con prioridad $N$ no puede acceder a recursos compartidos
  (conocidos como \textit{objetos protegidos o protected objects}) de prioridad
  inferior.
\item Si una tarea con prioridad $N$ accede a un objeto protegido
  de prioridad $M$
  con $M \geq N$, esta tarea adquiere la prioridad $M$ hasta que haya
  terminado de usar dicho objeto.
\item Una tarea no puede llamar a otra directamente.
\item Un objeto protegido solo puede tener una \textit{entry}
  \footnote{Procedimiento en el que una tarea va a quedarse esperando hasta que
  se dé una condición establecida en el objeto protegido.}.
\item Solo puede haber una tarea esperando en la \textit{entry} de un objeto
  protegido.
\item Todos los tiempos de espera (sentencias \textit{delay}) deben ser
  relativos.
\end{itemize}

En el caso de que no se cumpla alguna de las normas anteriores se lanzará
automáticamente la excepción PROGRAM\_ERROR finalizando la ejecución del
software.

Gracias al perfil de Ravenscar se consigue controlar de forma mucho más
eficiente la concurrencia, evitando entre otras cosas
la aparición de \textit{deadlocks}.

