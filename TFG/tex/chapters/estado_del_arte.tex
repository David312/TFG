
\chapter{Estado del Arte}
\label{chap:estado_del_arte}

El OBC del UPMSat-2 es un sistema empotrado (o embebido), ya que está
diseñado para cumplir un conjunto de necesidades específicas, y 
también un sistema de tiempo real por la necesidad de que responda a los
estímulos del entorno físico dentro de un intervalo de tiempo determinado.
Por ello, es necesario emplear herramientas que permitan cumplir con estas
necesidades de la manera más simple posible.
\nota{Es un sistema de alta integridad ya que no se puede reparar una vez en
  vuelo,por lo que tiene que ser diseñado muy cuidadosamente.}
Un lenguaje de programación que facilita esta tarea es Ada, ya que incorpora
de forma innata todos los mecanismos necesarios para el control de tiempo y la
interacción entre tareas o hilos de ejecución.
Además cabe destacar que el lenguaje Ada es
uno de los lenguajes oficiales de la Agencia Espacial Europea
(ESA\cite{web-ESA}) para este tipo de sistemas.

El software del OBC del UPMSat-2 por tanto ha sido desarrollado en el lenguaje
Ada usando un compilador cruzado\footnote{Compilador que genera un código
  ejecutable para otra arquitectura distinta.}
para procesadores de la arquitectura SPARC, en concreto el procesador LEON3.

Como cada uno de los subsistemas que componen el OBC tiene sus propias
tareas para llevar a cabo su funcionalidad, es necesario
controlar de alguna manera sus prioridades y la compartición de recursos entre
ellas.

Un ejemplo de la importancia de ajustar correctamente las prioridades de las
tareas es el siguiente:
Tenemos dos tareas: una para el control de la actitud del satélite (tarea
crítica) y otra para
el control de un experimento (tarea prescindible). Si la tarea de control de
actitud necesita
comprobar el estado del satélite cada medio segundo y resulta que ambas tareas
tienen la misma prioridad, puede ocurrir que la tarea del experimento esté
ocupando la CPU e impida cumplir con los requerimientos temporales de la
primera, dando posibilidad al descontrol del satélite y la pérdida de las
comunicaciones.

\nota{Hasta aquí las modificaciones.}\\

Para ello, se emplea el perfil de Ravenscar\cite{perfil-ravenscar},
que es un subconjunto de
las características de Ada que restringe ciertos aspectos como los siguientes:

\begin{itemize}
\item Una tarea con prioridad $N$ no puede acceder a recursos compartidos
  (conocidos como \textit{objetos protegidos o protected objects}) de prioridad
  inferior.
\item Si una tarea con prioridad $N$ accede a un objeto protegido
  de prioridad $M$
  con $M \geq N$, esta tarea adquiere la prioridad $M$ hasta que haya
  terminado de usar dicho objeto.
\item Una tarea no puede llamar a otra directamente.
\item Un objeto protegido solo puede tener una \textit{entry}
  \footnote{Procedimiento en el que una tarea va a quedarse esperando hasta que
  se dé una condición establecida en el objeto protegido.}.
\item Solo puede haber una tarea esperando en la \textit{entry} de un objeto
  protegido.
\item Todos los tiempos de espera (sentencias \textit{delay}) deben ser
  relativos.
\end{itemize}

En el caso de que no se cumpla alguna de las normas anteriores se producirá
una excepción que provocará el fin de la ejecución del software.

Gracias al perfil de Ravenscar se consigue controlar de forma mucho más
eficiente la concurrencia, evitando entre otras cosas
la aparición de \textit{deadlocks}.

