
\chapter{Introducción}
\label{chap:introduccion}

El satélite UPMSat-2 forma parte del proyecto UPMSat-2 UNION liderado por el
Institudo de Microgravedad Ignacio Da Rivas (IDR) junto con la Universidad
Politécnica de Madrid (UPM).

Como denota el nombre del satélite, este proyecto nace como sucesión al
ya lanzado exitosamente UPM-SAT1, y está siendo desarrollado en colaboración
con las distintas escuelas que componen la UPM, cada una aportando sus
conocimientos de ingeniería en los que está especializada.

Como detalles técnicos del satélite cabe destacar la información proporcionada
en la página web del proyecto\cite{web-upmsat2}:
\begin{itemize}
\item Peso: 50 kg.
\item Medidas: 0.5m x 0.5m x 0.6m (ancho, largo, alto).
\item Volumen útil: 0.4m x 0.4m x 0.25m.
\item Carga útil: 15 kg.
\item Potencia: 15 W.
\item Órbita polar: 600 km de altitud.
\item Vida útil: 2 años.
\end{itemize}

\section{Objetivos}
\label{sec:objetivos}

Este Trabajo de Fin de Grado (TFG) tiene como objetivo integrar el software (SW)
de vuelo del computador de a bordo (OBC\footnote{On Board Computer}) del
satélite.

Los objetivos concretos son los siguientes:
\begin{itemize}
\item Estudio y comprensión del SW de los distintos subsistemas ya
  desarrollados.
\item Análisis en profundidad del documento de requisitos software\cite{} (SRS
  \footnote{Software Requirements Specification}).
\item Desarrollo de un programa que permita ejecutar una batería de pruebas.
\item Desarrollo de un programa que permita diseñar las baterías de pruebas de
  forma gráfica.
\item Validación y verificación de los requisitos SW.
\item Preparación de parches para arreglar los fallos detectados en la
  validacion.
\end{itemize}


\section{Estructura del documento}
\label{sec:estructura doc}

Este documento está estructurado de la siguiente forma: en el capítulo
\ref{chap:estado del arte} se describe a grandes rasgos cuales son las
restricciones bajo las que se ha desarrollado el SW del OBC, sus necesidades
y cómo son las herramientas de desarrollo de las que se dispone.

\nota{A COMPLETAR CON EL RESTO DE CAPÍTULOS.}
