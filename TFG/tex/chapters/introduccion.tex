
\chapter{Introducción}
\label{chap:introduccion}

El satélite UPMSat-2 forma parte del proyecto UPMSat-2 UNION liderado por el
Institudo de Microgravedad Ignacio Da Rivas (IDR) junto con la Universidad
Politécnica de Madrid (UPM).

Como denota el nombre del satélite, este proyecto nace como sucesión al
ya lanzado exitosamente UPM-SAT1, y está siendo desarrollado por 
\nota{Mencionar solo los grupos de aeronáuticos, el DIT, etc.}

Como detalles técnicos del satélite cabe destacar la información proporcionada
en la página web del proyecto\cite{web-upmsat2}:
\begin{itemize}
\item Peso: 50 kg.
\item Medidas: 0.5m x 0.5m x 0.6m (ancho, largo, alto).
\item Volumen útil: 0.4m x 0.4m x 0.25m.
\item Carga útil: 15 kg.
\item Potencia: 15 W.
\item Órbita polar: 600 km de altitud.
\item Vida útil: 2 años.
\end{itemize}

\section{Equipamiento}
\label{sec:equipamiento}
Como equipamiento hardware el UPMSat-2 dispone de los siguientes componentes:
\begin{itemize}
\item Seis paneles solares (uno por cada eje X, Y, Z).
\item Seis sensores de temperaturas.
\item Una batería de \nota{averiguar capacidad}.
\item Tres magnetómetros para medir el campo magnético de la Tierra.
\item Tres magnetopares (uno por eje) para generar un campo magnético que permita
  mantener la orientación (también llamada actitud) del satélite en la
  posición adecuada.
\item Una tarjeta de radio para las comunicaciones con la estación de Tierra.
\item Un volante o rueda de inercia\footnote{Un volante de inercia se basa en el
    principio de conservación del momento angular para controlar la actitud de
    satélites. Su funcionamiento básico consiste en un motor que gira y genera
    un momento angular que es transmitido al satélite en el eje correspondiente.
    De esta forma pueden emplearse un conjunto de ruedas de inercia en lugar
    de usar magnetopares.} para experimentación.
\item Tres microswitches térmicos (microthermal switches) también para
  experimentación.
\item Un computador de a bordo (OBC\footnote{On Board Computer}) encargado de
  controlar y coordinar el funcionamiento mediante software de todos los
  dispositivos anteriores.
\end{itemize}
\section{Objetivos}
\label{sec:objetivos}

Este Trabajo de Fin de Grado (TFG) tiene como objetivo integrar el software (SW)
de vuelo del OBC.

Los objetivos concretos son los siguientes:
\begin{itemize}
\item Estudio y comprensión del SW de los distintos subsistemas ya
  desarrollados.
\item Análisis en profundidad del documento de requisitos software\cite{} (SRS
  \footnote{Software Requirements Specification}).
\item Desarrollo de un programa que permita ejecutar una batería de pruebas.
\item Desarrollo de un programa que permita diseñar las baterías de pruebas de
  forma gráfica.
\item Validación y verificación de los requisitos SW.
\item Arreglar los fallos SW detectados en la validación.
\end{itemize}


\section{Estructura del documento}
\label{sec:estructura doc}

Este documento está estructurado de la siguiente forma: en el capítulo
\ref{chap:estado_del_arte} se describe a grandes rasgos cuales son las
restricciones bajo las que se ha desarrollado el SW del OBC, sus necesidades
y cómo son las herramientas de desarrollo de las que se dispone.

\nota{A COMPLETAR CON EL RESTO DE CAPÍTULOS.}
