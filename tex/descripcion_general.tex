
\section{Descripción general del trabajo}
Mi Trabajo Final de Grado (TFG) consiste en la integración y validación del
software del computador de a bordo (OBC\footnote{Computador de a bordo u
  \textbf{\textit{On Board Computer}} en inglés.}) del satélite UPMSat-2.\\

  Para llevar a cabo esta tarea es necesario estudiar y aprender en detalle cómo
  funcionan los distintos subsistemas que componen el OBC así como la estructura
  de su software para poder corregir y depurar los errores.\\

  Cabe destacar que en la situación actual en la que me encuentro no existe un
  documento de especificación de requisitos software (SRS\footnote{Siglas de
    \textbf{\textit{Software Requirements Specification}} en inglés.}) definitivo, por lo
  que al trabajo hay que añadirle las reuniones con los ingenieros aeronáuticos
  del IDR/UPM puesto que ellos son los \textit{clientes} del software del OBC.\\

  Para poder verificar que el software integrado cumple con la SRS será
  necesario establecer un conjunto de pruebas de sistema a través del envío
  de telecomandos (TCs\footnote{Abreviatura de \textbf{\textit{TeleComandos}}.}) al OBC de manera que se pueda comprobar que éste
  responde correctamente a los mismos y que no surge ningún error software.

  Por ello, será necesario crear un \textit{framework} que facilite esta tarea.\\

  \subsection{Lista de objetivos}
  La lista de objetivos previstos es la siguiente:
  \begin{itemize}
  \item Estudiar y comprender los distintos subsistemas del OBC.
  \item Estudiar el diseño y organización del software del OBC.
  \item Aprendizaje de las herramientas con las que se trabajará.
  \item Estudio, revisión y comprensión de la SRS.
  \item Diseño de pruebas de sistema.
  \item Validación y verificación de los requisitos de la SRS.
  \item Depuración de errores encontrados.
  \item Integrar todo el software del OBC.
  \item Reunirse cada cierto tiempo con los clientes.
  \item Implementación de nuevos requisitos que puedan surgir o que hayan sido
    modificados.
  \end{itemize}