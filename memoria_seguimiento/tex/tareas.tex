
\section{Modificaciones en el plan de trabajo}
La lista de objetivos a lograr es la siguiente:

\begin{itemize}
\item Diseño de pruebas de sistema.
\item Validación y verificación de los requisitos de la SRS.
\item Depuración de nuevos errores encontrados.
\item Integrar todo el software del OBC, realizando las pruebas de
  verificación y validación.
\item Reunirse con los \textit{clientes}.
\item Implementación de nuevos requisitos que puedan surgir o que hayan sido
  modificados.
\end{itemize}

Los motivos principales de haber pospuesto estos objetivos han sido la falta
de tiempo por las tareas de otras asignaturas así como imprevistos que han
surgido a nivel personal.\\

Las reuniones con los clientes se han pospuesto por motivos fuera de mi alcance.

\section{Lista de tareas}
La lista de tareas por realizar son las siguientes:
\begin{itemize}
%   \itemsep0em
% \item Estudio de la documentación.
%   \begin{itemize}
%     \itemsep0em
%   \item Estudio en detalle de la funcionalidad de cada subsistema y la
%     interacción entre los mismos.
%   \item Análisis de la estructura del software del OBC y su diseño.
%   \item Estudio y análisis de la SRS.
%   \end{itemize}

% \item Desarrollo del framework de pruebas.
%   \begin{itemize}
%     \itemsep0em
%   \item Diseñar o buscar una estructura para poder escribir los TCs en un
%     fichero de texto.
%   \item Escribir un programa que trate dicho fichero.
%   \item Diseñar una aplicación gráfica que permita crear estos ficheros de
%     forma fácil, intuitiva y cómoda.
%   \end{itemize}

\item Integrar el software del OBC. (Al completo).

\item Probar que el software integrado cumple con la SRS.
  \begin{itemize}
    \itemsep0em
  \item Creación de ficheros de pruebas con pruebas de caja negra.
  \item Creación de ficheros de pruebas con pruebas de caja blanca.
  \item Observación de la respuesta del OBC a los TCs recibidos.
  \item Registro de bugs y comportamientos anómalos con respecto a lo
    especificado.
  \end{itemize}

\item Corrección de errores.  
  
\item Reunión con los clientes.
  \begin{itemize}
    \itemsep0em
  \item Mostrarles los avances conseguidos.
  \item Preguntarles sobre cómo actuar ante casos no previstos ni indicados
    en la SRS.
  \item Actualizar los resquisitos existentes y añadir nuevos requisitos.
  \end{itemize}

\item Llevar a cabo los cambios necesarios para cumplir con los cambios en la
  SRS.

\item Escritura de la memoria del TFG.
\item Preparación de la defensa del TFG.
  
\end{itemize}

