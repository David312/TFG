
\section{Propuesta de trabajo del tutor}

\paragraph{Título:}
Integración y validación del software del computador de a bordo del UPMSat-2.

\paragraph{Resumen del trabajo:}
UPMSat-2 es un proyecto de microsatélite universitario liderado por IDR/UPM y en
el que STRAST, nuestro grupo de investigación, se encarga del desarrollo del
software embarcado y del de la estación de tierra

(\url{http://www.idr.upm.es/tec\_espacial/upmsat2/01\_UPMSAT2.html}).

El trabajo consiste en la integración, pruebas de sistema y depuración de
errores de los distintos subsistemas que aportan la funcionalidad necesaria
para el correcto funcionamiento del segmento de vuelo.

Para ello se necesitan entrevistas con los ingenieros aeroespaciales a fin
de revisar el comportamiento final y el comportamiento ante fallos de hardware
y software.

\paragraph{Lista de objetivos concretos:}
\begin{itemize}
\item Estudio de los subsistemas y manejadores de dispositivos incluídos
  en el software.
\item Estudio del OBC (On-Board Computer) y de las herramientas de desarrollo,
  carga y depuración del software. Así como del sistema de validación
  del software.
\item Integración de todos estos subsistemas para generar el ejecutable
  del software.
\item Asignación de prioridades a las distintas tareas concurrentes para
  asegurar una ejecución correcta.
\item Desarrollo de un sistema de pruebas para pruebas de sistema.
\item Depuración de errores detectados y posibles desviaciones del
  comportamiento.
\item Desarrollo y adaptación de software a las nuevas necesidades.
\end{itemize}

\paragraph{Desglose de la dedicación total del trabajo en horas:}
\begin{itemize}
\item El alumno propuesto ha realizado el prácticum en el seno de este proyecto
  y posee conocimientos previos de desarrollo de software empotrado y tiempo
  real en el lenguaje de programación Ada.
\item No obstante, se estima en unas 100 horas la dedicación necesaria para
  aprender los requisitos del software de vuelo, sus distintos subsistemas y
  las herramientas necesarias para su integración y validación.
\item 50 horas para el desarrollo del sistema de pruebas que permita enviar
  órdenes y recibir telemetría del OBC de forma reproducible.
\item 65 horas de pruebas de integración, estudio de resultados e
  identificación de anomalías.
\item 60 horas para depurar los errores encontrados, entrevistas con
  el ``cliente'' (IDR/UPM) y realizar las modificaciones necesarias.
\item 49 horas para confeccionar la memoria y preparar la lectura.
\end{itemize}

\paragraph{Conocimientos prefios recomendados:}
Ingenería del software, programación y concurrencia, estructura de computadores,
sistemas empotrados, sistemas de tiempo real, entorno de desarrollo cruzado
de GNU sobre LinuX y lenguajes de programación Ada y C.